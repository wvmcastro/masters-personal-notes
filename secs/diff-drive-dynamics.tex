\documentclass[12pt]{article}
\usepackage{geometry}
     \geometry{
     a4paper,
     left=25mm,
     top=25mm,
     bottom=15mm,
     right=15mm
 }

\usepackage{myincludes}
\usepackage{import}

\begin{document}
\section{The differential drive model dynamics}
\subsection{The differential drive model}
In this section two dynamic models will be derived for the differential drive robot, and I will show which one is the most adequate and why.
A differential drive robot is a two-wheeled robot, the wheels are aligned and their velocities can be different from each other. When both wheels have the same velocity, the robot moves forward or backwards. If they have the same speed but in opposite directions, the robot spins around the wheels axle's midpoint, clockwise or counterclockwise. Any configuration different than these makes the robot follow a curved trajectory.  

 \begin{figure}[h]
    \centering
    \import{figs}{diff-drive}
    \caption{Differential drive robot schematic top view. The z-axis is pointing out of the sheet. The robot`s moving coordinates frame is placed in the middle point between the wheels axle. \{s\} is a static frame of reference.}
     \label{fig:diff-drive-schematic}
 \end{figure}

 \subsection{The differential drive kinematics}
 Let's first define the robot state by a state vector $\mathbf{x}$, Eq. \ref{eq:diff-drive-state-vector}, composed by the robot's heading and position.

 \begin{equation}
     \bvec{x_k} = \begin{bmatrix}
         \phi_\mathrm{k} \\ x_\mathrm{k} \\ y_\mathrm{k} 
     \end{bmatrix}
     \label{eq:diff-drive-state-vector}
 \end{equation}

 The goal is to derive a model that relates the wheel velocities (inputs), in \si[per-mode=symbol]{\radian\per\second}, to the vector representing the robot's state. Something with the shape of Eq. \ref{eq:diff-drive-generic-model}. Where $\mathbf{u}$ is the input vector.
 \begin{equation}
     \mathbf{\dot{x}} = \boldsymbol{g}\left( \begin{bmatrix}
         g_1(\mathbf{x}, \mathbf{u}) \\ g_2(\mathbf{x}, \mathbf{u}) \\ g_3(\mathbf{x}, \mathbf{u})\end{bmatrix} \right)
     \label{eq:diff-drive-generic-model}
 \end{equation}

 The input vector is the pair $(u_L, u_R)$, the angular velocities of the left and right wheels, respectively. The velocities signs are defined in a way that when both are positive, the robot moves in the x-axis positive direction of its own coordinate frame. The wheels linear velocities is given by Eq. \ref{eq:diff-drive-wheels-linear-velocities}, where $r$ is the wheels radius.

 \begin{equation}
 \begin{cases}
     \lVert \vec{\mathrm{v}}_L \rVert = u_L \cdot r\\
     \lVert \vec{\mathrm{v}}_R \rVert = u_R \cdot r
 \end{cases}
     \label{eq:diff-drive-wheels-linear-velocities}
 \end{equation}

 From now on, all the calculations will be made in the robot coordinates frame. The axle middle point is the robot position. One also must know the relation between linear and angular speeds in Eq. \ref{eq:angular-linear-speed}.

 \begin{equation}
     V = \omega R
     \label{eq:angular-linear-speed}
 \end{equation}

 \noindent where
 \begin{description}[labelindent=10pt, labelsep=10pt]
     \item[$V$] is the linear speed
     \item[$\omega$] is rotation speed, aka the angular speed
     \item[$R$] is the radius of rotation
 \end{description}

 First, I will derive the robot linear velocity in function of the wheels velocities. By Figure \ref{fig:diff-drive-schematic} it is clear that any wheel displacement will make the middle point translates in the $\hat{y}_\mathsmaller{R}$ direction, therefore we have that $\dot{y} = 0$. The linear velocity in the $\hat{x}_\mathsmaller{R}$ direction is given by Eq. \ref{eq:diff-drive-x-velocity},

 \begin{equation}
 \begin{aligned}
     \dot{x} &= \frac{1}{2} \,(\mathrm{v}_R + \mathrm{v}_L)\\
     &= \frac{r}{2} \,(u_R + u_L)
 \end{aligned}
     \label{eq:diff-drive-x-velocity}
 \end{equation}

 to see that, just observe that when one wheel is stopped and the other is spinning the middle point linear velocity will be half of the point in the wheel, because the middle point will describe a circle with half the radius of the circle described by the spinning wheel point around the stopped wheel. Then, just sum the contribution of each wheel.

 Now the angular velocity, just as with the linear velocity, I will analyse the contribution of each wheel and then, sum the effects. When the right wheel spins, the midpoint, $\boldsymbol{m}$ in Fig. \ref{fig:diff-drive-schematic}, rotates in the counterclockwise direction around the left wheel with linear velocity $\dfrac{\mathrm{v}_R}{2}$ and radius $d$. Therefore the angular velocity contribution of the right wheel is $\dot{\phi}_R = \dfrac{\mathrm{v}_R}{2\,d}$, in the counterclockwise direction. The contribution of the left wheel is analogous, but when it spins forward, the middle point moves clockwise around the right wheel, since I am adopting the right-handed coordinates system, the clockwise sense is negative and the counterclockwise is positive. The robot angular velocity is given by Eq. \ref{eq:diff-drive-angular-velocity}

 \begin{equation}
 \begin{aligned}
     \dot{\phi} &= \dot{\phi}_R + \dot{\phi}_L\\
     &= \frac{\mathrm{v}_R}{2\,d} - \frac{\mathrm{v}_L}{2\,d}\\
     &= \frac{r}{2\,d} \, (u_R - u_L)
 \end{aligned}
 \label{eq:diff-drive-angular-velocity}
 \end{equation}

 Wrapping everything matrix form, we end up with Eq. \ref{eq:conituous-diff-drive-model-in-robot-frame}

 \begin{equation}
     \begin{bmatrix}
         \dot{\phi} \\ \dot{x} \\ \dot{y}
     \end{bmatrix}_{\{\mathrm{R}\}}= \begin{bmatrix}
         -r/2d & r/2d \\
         r/2 & r/2 \\
         0 & 0 \\
     \end{bmatrix} \begin{bmatrix} u_L \\ u_R \end{bmatrix}
     \label{eq:conituous-diff-drive-model-in-robot-frame}
 \end{equation}

 Equation \ref{eq:conituous-diff-drive-model-in-robot-frame} is expressed in the robot frame, that is moving with it. To express it with respect to a static frame, as depicted in Fig. \ref{fig:diff-drive-schematic}, we need to write the linear velocities in terms of the static frame base vectors. Equation \ref{eq:robot-to-static-frame-transform} gives the basis change matrix from the robot frame to the static frame.

 \begin{equation}
     \boldsymbol{R_{\mathsmaller{\mathrm{SR}}}} = \begin{bmatrix}
         1 & 0 & 0 \\
         0 & \cos\phi & -\sin\phi \\
         0 & \sin\phi & \cos\phi \\
     \end{bmatrix}
     \label{eq:robot-to-static-frame-transform}
 \end{equation}

 One must remember that the z-axis of both frames is pointing out of the sheet, that's why the first column is $\begin{bmatrix} 1 & 0 & 0 \end{bmatrix}^T$. Using Eq. \ref{eq:conituous-diff-drive-model-in-robot-frame} and \ref{eq:robot-to-static-frame-transform} we obtain Eq. \ref{eq:continuous-diff-drive-model} which is the robot dynamic model in the static frame. Note that this is a nonlinear model, since we have sine and cosine function of the robot heading, one of the system's states.

 \begin{equation}
     \begin{aligned}
     \begin{bmatrix}
         \dot{\phi} \\ \dot{x} \\ \dot{y}
     \end{bmatrix}_{\{\mathrm{S}\}} &= \boldsymbol{R_{\mathsmaller{\mathrm{SR}}}} \begin{bmatrix}
          \dot{\phi} \\ \dot{x} \\ \dot{y}
     \end{bmatrix}_{\{\mathrm{R}\}} \\
     &= \begin{bmatrix}
         -r/2d & r/2d \\
         (r/2)\cos\phi & (r/2)\cos\phi \\
         (r/2)\sin\phi & (r/2)\sin\phi
     \end{bmatrix}  \begin{bmatrix} u_L \\ u_R \end{bmatrix}
     \end{aligned}
     \label{eq:continuous-diff-drive-model}
 \end{equation}

 To use this model with the Kalman Filter, we need to linearize it around 
 the predicted state and then discretize the model.

 \subsection{Another differential drive model}
 \renewcommand{\arraystretch}{1.5}
 \begin{equation}
     \mathbf{x_{k+1}} = \begin{bmatrix}
         \phi_k \\ x_k \\ y_k
     \end{bmatrix} + 
     \begin{bmatrix}
         \dfrac{r}{2d}\,\Delta\theta_{L_k} - \dfrac{r}{2d} \,\Delta\theta_{R_k} \\
         \dfrac{\Delta\theta_{R_k} + \Delta\theta_{L_k}}{\Delta\theta_{R_k} - \Delta\theta_{L_k}}\left( \sin(\phi_k + \alpha) - \sin(\phi_k) \right) \\
         \dfrac{\Delta\theta_{R_k} + \Delta\theta_{L_k}}{\Delta\theta_{R_k} - \Delta\theta_{L_k}}\left( -\cos(\phi_k + \alpha) + \cos(\phi_k) \right) 
     \end{bmatrix}
 \end{equation}
 \renewcommand{\arraystretch}{1}

\end{document}